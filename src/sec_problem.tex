\section{Modeling of Thermal Protection Systems}

This section presents the problem of modeling the transient thermo-ablative response of a non-decomposing TPS, subjected to extreme hypersonic heating. Two different but mathematically connected solution strategies are provided: (1) a high-fidelity full-order model (FOM) based on a finite element method (FEM), and (2) a RPM based on a \textit{lumped capacitance model} (LCM) coupled with a one-dimensional \textit{surface recession model} (SRM). The FOM is computationally expensive but provides the highest fidelity, while the RPM is computationally efficient but has low predictive fidelity. However, both models are physically consistent to high-dimensional design variables. The following discussion presents the TPS modeling problem and the FOM and RPM solution strategies.

\subsection{Governing Equations}\label{sec_governing_equations}
The multi-physics of a non-decomposing ablating TPS under a hypersonic boundary layer involves the \textit{energy equation} for heat conduction inside the TPS, and the \textit{pseudo-elasticity equation} for mesh motion due to surface recession. The coupling between these two equations occurs at the heated boundary, where the surface temperature drives the surface recession velocity, which appears as an advection term in the energy equation. The governing PDEs are described as follows.

\subsubsection{Energy Equation}
Consider a generic domain $\Omega\subset\mathbb{R}^d$, $d=2$ or $3$, illustrated in Fig.~\ref{fig_general_domain}. Let $\partial\Omega = \Gamma_q\cup\Gamma_T$ and $\Gamma_q\cap\Gamma_T = \emptyset$, where a Neumann $q_b(x,t)$ boundary condition is prescribed on the heated boundary $\Gamma_q$, and represents the surface exposed to the hypersonic boundary layer. The Dirichlet $T_b(x,t)$ boundary condition is prescribed on the boundary $\Gamma_T$. The TPS is divided into $N$ non-overlapping components $\left\{\Omega_i\right\}_{i=1}^{N}$, as illustrated in Fig.~\ref{fig_general_domain} for $N=2$. The $i$-th component $\Omega_i$ is associated with material properties $\left(\rho_i,c_{p,i}, \vk_i\right)$, which are continuous within one component, and can be discontinuous across two neighboring components.

\begin{figure}
    \centering
    \includegraphics[width=0.6\textwidth]{./figs/general_domain.png}
    \caption{General domain $\Omega$ with prescribed Neumann and Dirichlet boundary conditions on $\Gamma_q$ and $\Gamma_T$. Mesh displacement $w(x,t)$ occurs on the $\Gamma_q$ boundary.}
    \label{fig_general_domain}
\end{figure}

The energy equation describes the transient heat conduction,
\begin{subequations}
    \begin{align}
        \rho c_p\left(\ppt{T} + \tilde{\vv}(x,t)\cdot\nabla T\right) - \nabla\cdot (\mathbf{k}\nabla T) &= 0,\ x\in\Omega \label{eqn_thermal_pde}\\
        -\mathbf{k}\nabla T\cdot \vn &= q_b(x,t),\ x\in\Gamma_q\label{eqn_thermal_bc_neumann}\\
        T(x,t) &= T_b(x,t),\ x\in\Gamma_T\label{eqn_thermal_bc_dirichlet}\\
        T(x,0) &= T_0(x),\ x\in\Omega\label{eqn_thermal_ic}
    \end{align}\label{eqn_governing_equations}
\end{subequations}
where $\rho$, $c_p$, and $\mathbf{k}\in\mathbb{R}^{d\times d}$ are the constant density, heat capacity, and thermal conductivity. Note that our prior work has applied the PIROM to TPS problems with temperature-varying material properties~\cite{VargasVenegas2025}. In the order they appear, the $\rho c_p\frac{\partial T}{\partial t}$ term includes the unsteady energy storage, $\rho c_p\tilde{\vv}(x,t)\cdot\nabla T$ includes the temperature advection due to ablation, and $\nabla\cdot(\mathbf{k}\nabla T)$ includes the heat conduction.

An Abirtrary Lagrangian-Eulerian (ALE) description is used to account for mesh motion due to surface recession. The relative velocity of the material $\tilde{\vv}(x,t)$ with respect to the mesh is,
\begin{equation}
    \tilde{\vv}(x,t) = \vv_s(x,t) - \vv_m(x,t)\label{eqn_relative_velocity}
\end{equation}
where $\vv_s(x,t)$ and $\vv_m(x,t)$ are the physical material velocity and mesh velocity, respectively. In this work, the physical material velocity is assumed to be zero, i.e., $\vv_s(x,t)=\vzero$, and thus the relative velocity is simply the negative of the mesh velocity, $\tilde{\vv}(x,t) = -\vv_m(x,t)$.

\subsubsection{Pseudo-Elasticity Equation}
The mesh displacements $\vd\in\mathbb{R}^d$ are described by the steady-state pseudo-elasticity equation, which models the mesh as a fictitious elastic solid that deforms according to the prescribed boundary displacements. The governing equation is given as,
\begin{subequations}
    \begin{align}
        \nabla\cdot\vsigma(\vd) &= \vzero,\quad \forall\:t\in\cT,\:x\in\Omega\quad\label{eqn_elasticity_pde}\\
        \vd(x,t) &= \vd_q(x,t),\quad \forall\:t\in\cT,\: x\in\Gamma_q\label{eqn_displacement_heated_bc}\\
        \vd(x,t) &= \vzero,\quad \forall\:t\in\cT,\: x \in \Gamma_T\label{eqn_displacement_unheated_bc}\\
        \vd(x,0) &= \vzero,\quad \forall\:x\in\Omega\label{eqn_displacement_initial_condition}
    \end{align}
\end{subequations}
where the stress tensor $\vsigma$ is related to the strain tensor $\vepsilon(\vd)$ through Hooke's law,
\[
    \vsigma(\vd) = \mathbb{D}:\vepsilon(\vd)
\]
where $\mathbb{D}$ is the fourth-order positive definite elasticity tensor, and ``:'' is the double contraction of the full-order tensor $\mathbb{D}$ with the second-order tensor $\vepsilon$. The elasticity tensor ordinarily possess a number of symmetries, effectively reducing the number of components that describe it~\cite{Bergel2022}. In this work, the standard isotropic case with rotational symmetry is considered, where $\mathbb{D}$ is fully described by two Lam\'e parameters $\lambda$ and $\mu$ arbitrarily chosen to tailor the mesh deformation. The symmetric strain tensor $\vepsilon$ measures the deformation of the mesh due to displacements $\vd(x,t)$, and is defined as,
\[
    \vepsilon(\vd) = \frac{1}{2}\left(\nabla\vd + \nabla\vd^\top\right)
\]
The ``material'' properties for the mesh are chosen to tailor the mesh deformation, and need not represent the actual material being modeled~\cite{Amar2016}.

For the pseudo-elasticity equations, the boundary conditions include prescribed displacements $\vd_q(x,t)$ on the heated boundary $\Gamma_q$ in \cref{eqn_displacement_heated_bc}, and zero displacements on the unheated boundaries in \cref{eqn_displacement_unheated_bc}. The initial condition for the mesh displacements is zero in \cref{eqn_displacement_initial_condition}. Particularly, the surface velocity due to the ablating material is a function of the surface temperature $T_q(x,t)$ for $x\in\Gamma_q$ on the heated boundary. For the $i$-th material component, the mesh velocity on the heated boundary is imposed based on the following relation,
\begin{equation}
    \hat{\vn}\cdot\vv_m(x,t) = f_i(T_q(x,t)),\quad x\in\Gamma_{q,i}\label{eqn_boundary_velocity}
\end{equation}
where $\Gamma_{q} = \cup_{i=1}^{\tilde{N}}\Gamma_{q,i}$ with $\Gamma_{q,i}$ as the portion of the heated boundary that belongs to the $i$-th ablative component, $\tilde{N}$ is the number of ablative components with $\tilde{N}\leq N$, $\hat{\vn}$ is the unit normal vector, and $f_i$ is a material-dependent function obtained from tabulated data, commonly referred to as a B' table~\cite{Amar2016}. The B' table provides a model for the recession velocity as a function of the surface temperature, and is pre-computed based on high-fidelity simulations or physical experiments for a one-dimensional slab of materials, and is independent of the TPS geometry. Provided the surface velocity, the boundary condition in \cref{eqn_boundary_displacement} for the mesh displacements are computed by integrating the surface velocity over time,
\begin{equation}
    \vd_q(x,t) = \int_{0}^{t} \mathbf{v}_m(x,\tau)d\tau\label{eqn_boundary_displacement}
\end{equation}

\subsection{Full-Order Model: Finite-Element Method}\label{sec_fom}

The following discussion presents the high-fidelity transient thermo-ablative modeling of the TPS as the FOM. Specifically, the FOM is based on \textit{finite-element methods} (FEM) for the governing PDEs in \cref{eqn_governing_equations}, and is implemented in the SIERRA/Aria code developed at Sandia National Laboratories~\cite{Bergel2022}. A \textit{Discontinuous Galerkin FEM} (DG-FEM) method is used to spatially discretize the energy equation for theoretical convenience, while a standard Galerkin FEM is used to spatially discretize the pseudo-elasticity equation for mesh motion. The following discussion presents the spatial discretization of both governing equations.

\paragraph*{Energy Equation} To obtain the full-order numerical solution, the \textit{energy equation} is spatially discretized using variational principles of DG-FEM~\cite{Cohen2018}. Note that the choice of DG approach is mainly for theoretical convenience, and is exclusively performed on the energy equation, as it is the surface temperature that drives the ablation process; the equivalence between DG and FEM is noted upon their convergence. Consider a conforming mesh partition domain, where each element belongs to one and only one component. Denote the collection of all $M$ elements as $\left\{E_i\right\}_{i=1}^{M}$. In an element $E_i$, its shared boundaries with another element $E_j$, Neumann BC, and Dirichlet BC are denoted as $e_{ij}$, $e_{iq}$, and $e_{iT}$, respectively. Lastly, $\left|e\right|$ denotes the length $(n_d=2)$ or area $(n_d=3)$ of a component boundary $e$. For the $i$-th element, use a set of $P$ trial functions, such as polynomials, to represent the temperature distribution,
\begin{equation}
    T_i(x,t) = \sum_{l=1}^{P} \phi_l^i(x) u_l^i(t) \equiv \vphi_i^\top(x)\vu_i(t),\quad i=1,2,\dots,M\label{eqn_element_temperature}
\end{equation}
Then, the energy equation is collected into a block-system of ODEs for all the elements in the mesh,
\begin{equation}
    \vA\dot{\vu} = \left[\vB + \vC(\vu)\right]\vu + \mathbf{f}(t)\label{eqn_full_dg}
\end{equation}
where $\vu = \left[\vu_1, \vu_2, \ldots, \vu_M\right]^\top\in\mathbb{R}^{MP}$ includes all the DG variables, $\mathbf{f}\in\mathbb{R}^{MP}$ is the external forcing, and the system matrices $\vA$, $\vB$, and $\vC$ are the matrices due to heat capacity, heat conduction, and temperature advection due to mesh motion, respectively. Note that the advection matrix $\vC(\vu)$ is a function of the temperature $\vu$ since it depends on the mesh velocity as in \cref{eqn_boundary_velocity}; this is the main source of non-linearity in the current TPS problem. A detailed derivation of \cref{eqn_full_dg} and their matrices is provided in Appendix~\ref{appendix_mathematical_details}.

\paragraph*{Pseudo-Elasticity Equation} The \textit{pseudo-elasticity equation} is spatially discretized using the standard Galerkin FEM method on a structured mesh with quadrilateral elements. Define the scalar basis functions $\left\{\psi_q(x)\right\}_{q=1}^{Q}$ nodal variables $\left\{\vd_q\right\}_{q=1}^{Q}$ for the mesh displacements, where $Q$ is the number of basis functions. Express the mesh displacements $\vw$ at time $t$ as,
\begin{equation}
    \vw(x,t) \approx \sum_{q=1}^{Q} \psi_q(x) \vw_q\label{eqn_mesh_fem_approximation}
\end{equation}
Substituting into the Galerkin weak form of the \textit{steady} pseudo-elasticity equation, the following linear system of equations is obtained for the nodal displacements,
\begin{equation}
    \vK\vw = \vg\label{eqn_fem_elasticity}
\end{equation}
where $\vw$ is the global displacement vector, $\vK\in\mathbb{R}^{dQ\times dQ}$ the global stiffness matrix of dimension $d$, defined by the volume integrals over the domain $\Omega$ provided the elasticity tensor $\mathbb{D}$, and $\vg$ is the global force vector due to the Dirichlet boundary conditions on the heated $\Gamma_q$ and unheated $\Gamma_T$ boundaries.

\subsection{Reduced-Physics Model}
The RPM for predicting the response of the ablating TPS consists of two components: (1) \textit{surface recession model} (SRM) and a \textit{lumped capacitance model} (LCM). The SRM provides a relation between the surface temperature and \textit{one-dimensional} surface recession velocity based on pre-computed B' tables for the material, enabling the computation of \textit{one-dimensional} surface displacements. Provided the geometry changed induced by the surface recession, the LCM predicts the average temperature inside each component of the TPS, which are in turn used as low-fidelity estimates for the surface temperatures required by the SRM. Therefore, the SRM and LCM are coupled to define the RPM, providing low-fidelity estimates for temperatures and surface recessions of the ablating TPS.

\subsubsection{Surface Recession Model}

The mesh displacements $\vd$ are constrained to be \textit{one-dimensional} on the heated boundary $\Gamma_q$, i.e., $w_i(x,t) = \vd(x,t)\cdot\hat{\vn}_i$, where $\hat{\vn}_i$ is the unit normal vector on the heated boundary $\Gamma_{q,i}$. Displacements perpendicular to $\hat{\vn}_i$ for $i=1,\dots,\tilde{N}$ are assumed to be small and thus neglected. Let $\vw = \left[w_1,w_2,\dots,w_{\tN}\right]^\top\in\mathbb{R}^{\tN}$ include the one-dimensional displacements for the $\tN$ ablating components on the heated boundary, where $\tN\leq N$. Then the SRM is described as,
\begin{equation}
    \dot{\vw} = \vXi\vu - \tvf\label{eqn_srm}
\end{equation}
where $\vXi=\text{diag}\left(\alpha_1,\dots,\alpha_{\tN}\right)$ and $\tvf=\left(\alpha_1 u_{0,1},\dots,\alpha_{\tN} u_{0,\tN}\right)^\top$. The constants $\alpha_i$ are small material-dependent parameters, determined from the B' table, and $u_{0,i}$ is the constant initial temperature of the ablative component. The SRM provides a relation between the surface's temperature and recession velocity, based on pre-computed B' tables for the material.

\subsubsection{Lumped Capacitance Model}

A general form of the LCM is provided in this section; details regarding the derivation for the four-component TPS used in the results section are provided in Appendix~\ref{appendix_mathematical_details}. Let $\Omega$ be partitioned into $N$ non-overlapping components $\left\{\Omega_i\right\}_{i=1}^{N}$, as illustrated in Fig.~\ref{fig_general_domain} for $N=2$. The domain $\Omega$ is a function of the surface displacements $\vw$, and thus the geometry of each component $\Omega_i$ is time-dependent. The LCM predicts the temporal variation of average temperatures in multiple shape-varying interconnected components~\cite{Incropera2011}. From a point of sview of energy conservation, the LCM leads to the following system of first-order ODEs for the average temperatures in the components,
\begin{equation}
    \bar{\vA}(\vw)\dot{\bar{\vu}} = \bar{\vB}(\vw)\bar{\vu} + \bar{\vf}(t)\label{eqn_lcm}
\end{equation}
Where the states and inputs,
\begin{equation}
    \bvu = \left[\bar{u}_1,\bar{u}_2,\dots,\bar{u}_N\right]^\top\in\mathbb{R}^{N},\quad \bar{\vf} = \left[\bar{f}_1,\bar{f}_2,\dots,\bar{f}_N\right]^\top\in\mathbb{R}^{N}\label{eqn_lcm_states_inputs}
\end{equation}
include the average temperatures $\bar{\vu}$ and spatially-integrated inputs $\bar{\vf}$ for the $N$ components. For $i,j=1,2,\dots,N$ the $(i,j)$-th elements of the $\bar{\vA}\in\mathbb{R}^{N\times N}$, $\bar{\vB}\in\mathbb{R}^{N\times N}$, and $\bar{\vf}\in\mathbb{R}^{N}$ matrices are given by,
\begin{subequations}
    \begin{gather}
        \bar{A}_i = \begin{cases}
                \int_{\Omega_i}\rho c_p d\Omega_i, & i=j\\
                0, & i\neq j
            \end{cases},\quad \bar{B}_{ij} = \begin{cases}
            \sumneighbordirichlet\bar{B}^i_{ij}, &i=j \\
            \bar{B}^{(j)}_{ij}, & i\neq j
        \end{cases},\\ \vf_i = \begin{cases}
            |\eiq|\bar{q}_i + \frac{|\eiT|}{R_{i}}\bar{T}_i, & i=j \\ 
            0, & i\neq j
        \end{cases}
    \end{gather}\label{eqn_lcm_matrices}
\end{subequations}
where,
\begin{equation}
    \bar{q}_i = \frac{1}{|\eiq|}\int_{\eiq} q_b d\eiq,\quad \bar{T}_i = \frac{1}{|\eiT|}\int_{\eiT}T_b d\eiT,\quad \bar{B}^{i}_{ij} = -\frac{|\eij|}{R_{ij}},\quad \bar{B}_{ij}^{j} = \frac{|\eij|}{R_{ij}}\label{eqn_lcm_matrices_elements}
\end{equation}
where $R_{ij}$ is the equivalent thermal resistance between two neighboring components $\Omega_i$ and $\Omega_j$, and $R_i$ is the thermal resistance between component $\Omega_i$ and the Dirichlet boundary. Note that the heat capacitances and thermal resistances are computed based on the current geometry of each component, which is a function of $\vw$ provided by the SRM.

\subsubsection{Thermo-Ablative Reduced-Physics Model}

The SRM and LCM are combined to define the RPM for predicting the thermo-ablative response of the TPS under hypersonic boundary layers. Specifically, the RPM is defined as the LCM as in \cref{eqn_lcm}, where the \textit{geometry-dependent} matrices $\bar{\vA}$ and $\bar{\vB}$ are updated at each time step based on the current displacements $\vw$ provided by the SRM. The RPM is formally stated as,
\begin{subequations}
    \begin{align}
        \tvA(\vs)\dot{\vs} &= \tvB(\vs)\vs + \tvF(t)\label{eqn_rpm}\\
        \tilde{\vz} &= \vs\label{eqn_rpm_observable}
    \end{align}
\end{subequations}
where the state $\vs=\left[\bvu,\vw\right]^\top\in\mathbb{R}^{N+\tN}$ includes the \textit{average temperature} and \textit{one-dimensional surface displacements}, and $\tN$ is the number of ablating components with $\tN\leq N$. Moreover, the observables are defined as $\vz=\left[\bvu,\vw\right]^\top\in\mathbb{R}^{N+\tN}$. The matrices are given as,
\begin{equation}
    \tvA(\vs) = \begin{bmatrix}
        \bar{\vA}(\vs) & \vzero\\
        \vzero & \vI
    \end{bmatrix},\quad \tvB(\vs) = \begin{bmatrix}
        \bar{\vB}(\vs) & \vzero\\
        \vXi & \vzero
    \end{bmatrix},\quad \tvF(t) = \begin{bmatrix}
        \bar{\vf}(t)\\
        -\tvf
    \end{bmatrix}\label{eqn_rpm_matrices}
\end{equation}
In the matrices $\tvA$ and $\tvB$, the surface displacements $\vw$ are used to define the dimensions for the $\Omega_i$ component used in \cref{eqn_lcm_matrices,eqn_lcm_matrices_elements}, thus effectively coupling the LCM and SRM.

\subsection{Summary of Modeling Approaches}
The FOM (i.e., FEM) and RPM (i.e., LCM with SRM) are two different but mathematically connected solution strategies. Particularly, the LCM in \cref{eqn_lcm} not only resembles the functional form of the DG model in \cref{eqn_full_dg}, but can be viewed as a special case of the latter, where the mesh partition is extremely coarse, and the trial and test functions are piece-wise constants. This removes all spatial variations within each component, and neglects advection effects due to mesh motion.

For example, consider the case where each component $\Omega_i$ is treated as one single element, and each element employs one constant basis function $\phi_i=1$. The DG-FEM model for the $i$-th component simplifies to the scalar ODE,
\begin{equation}
    \vAi = \bar{A}_i,\quad \vCi = 0, \quad\vBiij = -\sigma|\eij|,\quad \vBjij=\sigma|\eij|,\quad \vf_i = |\eiq|\bar{q}_i + \sigma|\eiT|\bar{T}_i
\end{equation}
Clearly, the LCM is a coarse zeroth-order DG model with the inverse of thermal resistance chosen as the element-wise penalty factors. Or conversely, the DG model is a refined version of LCM via \textit{hp}-adaptation.

The FOM and RPM represent two extremes in the modeling fidelity and computational cost spectrum. On one hand, the FOM is the most accurate but computationally expensive to evaluate due to the fine mesh discretizations for both the temperature and displacement fields, leading to possibly millions of state variables. On the other hand, the RPM considers only the average temperature of the material, from which the displacements are obtained by integrating the velocity. The coarsened representation of the temperature field significantly reduces the number of state variables to only a few per component, and thus reducing the computational cost. However, this sacrifices local temperature information that becomes critical to properly capture higher-order effects due to mesh motion and thermal gradients within each component. Thus, neither the FOM nor the RPM is an universal approach for real-world analysis, design, and optimization tasks for ablating TPS, where thousands of high-fidelity model evaluations may be necessary. This issue motivates the development of the PIROM, which can achieve the fidelity of FOM at a computational cost close to the RPM, while maintaining the generalizability to model parameters.