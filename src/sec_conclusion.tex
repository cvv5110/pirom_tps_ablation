\section{Conclusions}
This work presents the development and validation of the \textit{scientific machine learning} framework termed \textit{Physics-Informed Reduced Order Model} (PIROM) for simulating the transient thermo-ablative response of hypersonic thermal protection systems (TPS) subjected to hypersonic boundary layers. Using coarse-graining on a DG-FEM model and the Mori-Zwanzig formalism, the PIROM formulation in Ref.~\cite{VargasVenegas2025} is extended to account for non-decomposing thermo-ablative response of a multi-layered TPS. The PIROM builds upon the following two key components: (1) a first-order physics-based model, i.e., the RPM based on LCM and SRM, for low-fidelity predictions of the surface temperature and recession; and (2) a data-driven closure to the non-Markovian term in the Generalized Langevin Equation (GLE). The non-Markovian closure is recast as a set of hidden states that evolve according to a data-driven dynamical system that is learned from a sparse collection of high-fidelity temperature signals.

The results demonstrate that the PIROM framework effectively reconciles the trade-offs between accuracy, generalizability, and efficiency of the ITM for simulating ablating hypersonic TPS. The PIROM consistently achieves mean observable prediction errors of less than $1\%$ for extrapolative settings involving time-and-space varying boundary conditions and SRM models. Notably, the PIROM improves the RPM's accuracy by an order of magnitude while preserving its computational efficiency, physical interpretability, and parametric generalizability. Moreover, the PIROM delivers online evaluations that are two orders of magnitude faster than the FOM. These results highlight the PIROM's potential as a promising framework for optimizing multi-physical dynamical systems, such as TPS under diverse operating conditions.