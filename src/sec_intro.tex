\section{Introduction}
At hypersonic speeds, aerospace vehicles experience extreme aero-thermo-dynamic environments that require specialized thermal protection systems (TPS) to shield internal sub-structures, electronics, and possibly crew members from the intense aerodynamic heating. The TPS is composed of ablating materials to withstand the high-energy physics -- a high-temperature-capable and fibrous material injected with a resin that fills the pore network and strengthens the composite~\cite{Amar2016}. The TPS design promotes the exchange of mass through thermal and chemical reactions (i.e., ablation), effectively mitigating heat transfer to the sub-structures. As a result, accurate prediction for the ablating TPS response under extreme hypersonic heating becomes critical to ensuring survivability, performance, and safety of hypersonic vehicles.

Even with today's advancements in computational resources and numerical methods, high-fidelity simulations of ablating TPS remains a formidable challenge, both theoretically and computationally. On the theoretical side, the thermo-chemical reactions, coupled with the irregular pore network structure and ablating boundaries, translate into complex non-linear equations governing multi-physical interactions across several spatio-temporal scales~\cite{Amar2016,Howard2015}. On the computational side, numerical approaches based on finite-element (FEM) or finite-volume (FVM) methods yield systems of differential equations modeling the transient thermo-ablative response of the TPS~\cite{Cohen2018}. The FEM discretizations lead to high-dimensional systems of equations, resulting in prohibitive computational costs for many-query applications such as design, optimization, uncertainty quantification, and real-time applications, where possibly thousands of model evaluations are required.

Reduced-order models (ROMs) have emerged as a promising approach to alleviate the computational costs of high-fidelity simulations~\cite{Fang2024,Neelan2025}. Ideally, a ROM should be: (1) accurate to reproduce high-fidelity solutions, (2) support continuous or infinite-dimensional design parameters such as geometrical shapes and material distributions, (3) be computationally efficient to evaluate to allow for fast turnaround times in design optimization. However, the above three capabilities usually form an \textit{impossible trinity of modeling}, as illustrated in Fig.~\ref{fig_itm}; building a ROM that achieves any two capabilities sacrifices the third.

\begin{figure}
    \centering
    \includegraphics[width=0.6\linewidth]{./figs/trinity.png}
    \caption{The impossible trinity of modeling: accuracy, generalizability, and efficiency.}
    \label{fig_itm}
\end{figure}

The impossible trinity poses a significant challenge in the development of ROMs for the multi-disciplinary transient analysis and optimization of ablating TPS. Specifically, full-order models (FOMs), e.g., FEMs or FVMs, offer high accuracy and robust generalization over design spaces, but are computationally expensive to evaluate. Reduced-physics models (RPMs) -- such as simplified convective-radiative heat transfer or engineering correlations -- are low-dimensional models that achieve efficiency and broad applicability by ignoring higher-order non-linear effects. However, RPMs sacrifice accuracy for complex thermo-ablative responses due to the simplifications and assumptions inherent in their formulation, and it is generally not clear how to systematically leverage existing high-fidelity data to improve RPMs~\cite{Xian2021}.

Lastly, data-driven ROMs, such as Gaussian Process Regression (GPR)~\cite{Rasmussen2006}, Neural Networks (NNs), and neural ordinary differential equations (NODEs)~\cite{Chen2019}, can provide accurate and computationally-efficient approximations of high-fidelity models for complex thermo-ablative responses. However, these data-centric approaches often demand extensive high-fidelity data for training, do not necessarily satisfy fundamental physical constraints or conservation laws, and thus do not generalize well to the design spaces outside the training~\cite{Thelen2022}. For example, our previous work demonstrated that NODEs trained on high-fidelity data of non-ablating TPS failed to generalize when subjected to boundary conditions and material models outside the training set~\cite{VargasVenegas2025}.

This work presents the extension of the \textit{physics-infused reduced-order modeling} (PIROM) framework to include effects of ablation for TPS applications, previously ignored in Ref.~\cite{VargasVenegas2025}. Specifically, the PIROM is demonstrated for the transient thermo-ablative response of multi-layered hypersonic TPS. The PIROM is a non-intrusive framework that combines the strengths of physics-based models with machine learning to formulate and train ROMs for parametrized non-linear dynamical systems. The backbone of the PIROM is the physics-based component, i.e., the RPM, which in this work is composed of: (1) a \textit{lumped capacitance model} (LCM) to model the average heat transfer within the TPS layers, and (2) a \textit{surface recession model} (SRM) to model one-dimensional surface ablation. 

Leveraging the \textit{Mori-Zwanzig} (MZ) formalism~\cite{Duraisamy2025,Parish2017,Parish2017a}, the RPM is rigorously extended with data-driven hidden dynamics to account for the missing physics in the LCM, which are learned from high-fidelity data. The hidden dynamics enable higher predictive accuracy of the PIROM when subjected to complex boundary conditions and SRM model variations. For the TPS problem, the MZ approach produces a sufficiently simple model form while maintaining the physical consistency of the PIROM, as well as the dependence on design parameters. Thus, the PIROM aims to solve the ITM by leveraging the generalizability and computational efficiency of RPMs, while incorporating the accuracy and adaptability of data-driven extensions. More importantly, the PIROM formulation provides a general methodology for developing PIROMs for other multi-physics problems.

The specific objectives of this work are summarized as follows:
\begin{enumerate}
    \item Extend the previous PIROM formulation in Ref.~\cite{VargasVenegas2025} to model transient thermo-ablative response of multi-layered hypersonic TPS through a systematic coarse-graining procedure based on the Mori-Zwanzig formalism.
    \item Benchmark the accuracy, generalizability, and computational accelerations of the PIROM against the RPM and the high-fidelity FEM solutions of ablating TPS, thus quantifying the PIROM's capabilities to solve the ITM in complex multi-physical non-linear dynamical systems.
\end{enumerate}

